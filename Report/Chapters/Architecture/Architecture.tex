\chapter{Methods}
\label{chap:projectdef}



\begin{wrapfigure}{R}{0.3\textwidth}
	\centering
	\includegraphics[width=0.6\textwidth]{Figures/Architecture/Project_framework_diagram}
	\caption{The framework for the system used in this project}
	\label{fig:Project_framework}
\end{wrapfigure}
This project is about learning a car or robot to control and navigate it self. This should be done so the robot don't hit walls or obstacles. to solve this problem, a framework is created, the framework can be seen on \Cref{fig:Project_framework}.

The agent takes information from the environment, in this project the information used as input to the agent is an image. The agent uses the pixels from the image, to process spatial dependencies. The spatial dependencies is found by using of convolutional neural networks (CNN). The output from the CNN is the image features. The image features is used to process the temporal dependencies. To find the temporal dependencies a Recurrent Neural Network (RNN) is used. The RNN used is a special RNN called Long Short Term Memory network (LSTM). When both the spatial and temporal dependencies are found, the state the environment is found. This state is used to build the value function and a policy, which is describe a the Q-network. After the value function and the policy is found, the best suited action to the environment can be determined.


The framework is created with inspiration from the papers  \cite{DBLP:journals/corr/MnihBMGLHSK16} and \cite{Sallab:2017:2470-1173:70}
\newline




%\begin{figure}[H]
%	\centering
% \includegraphics[width=0.3\textwidth]{Figures/ProjectFramework/Project_framework_diagram}
%	\caption{The framework for the system used in this project}
%	\label{fig:Project_framework}
%\end{figure}

\section{Deep Q Networks}
\label{sec:DQN}
The first deep reinforcement learning network there has been created in this project is the Deep Q Network. The Deep Q Network was first time described in the paper "playing Atari with Deep reinforcement learning" \cite{DBLP:journals/corr/MnihKSGAWR13} published by DeepMind. In this paper they learned the computer to play Atari 2600 video games. The computer only observed the screen pixels from the game, and received a reward when the game score increased. The result was remarkable, because the games and the goals in the games are very different. The same architecture was used for seven different games. In 3 of the seven games the computer performed better than the best human. 

To understand the problem the Deep Q Network solved, it is easier to use an example, here we use the game Breakout as an example. In this game you control a paddle at the bottom of the screen, and have to clear the bricks in the top of the screen. This is done by bouncing a ball between the paddle and the bricks. Each time a brick is hit it disappears and the game score increases. The game can be seen on figure \Cref{fig:Atari_breakout}

\begin{figure}[H]
	\centering
	\includegraphics[width=1\textwidth]{Figures/Architecture/DQN/Atari_breakout.png}
	\caption{Atari Breakout game. Image credit: DeepMind\cite{DBLP:journals/corr/MnihKSGAWR13} }
	\label{fig:Atari_breakout}
\end{figure}

To teach a Network how to play this game, the input to the network is the screen images, and the output is the actions of the game: left, right or fire (to launch the ball). One approach to this problem is to treat it as a classification problem - for each screen decide if the paddle should move left, right or press fire. To do this lots of training examples is needed. The problem about this approach is thats really how human learns, we don't need someone to tell us million times which move to choose at each screen. Instead human just need occasional feedback that we did the right thing, and then learn from it ourself. This is the task reinforcement learning tries to solve, more of the reinforcement learning theory can be read in \textbf{Reference to Theory chapter}.

While the idea is quite intuitive, in practice there are numerous challenges. One of the challenges is when hitting a brick and the reward is received, it often has nothing to do with the actions (paddle movement) just before the reward was received. All the work was done when the paddle was positioned correctly and bounced the ball back. This is called the credit assignment problem – i.e., which of the preceding actions was responsible for getting the reward and to what extent.

Another challenge is when a strategy is found and a certain reward is received, should the program stick with that strategy or experiment with something that could lead to a bigger reward. This is called the explore-exploit dilemma – should you exploit the known working strategy or explore other, possibly better strategies. 

The screen pixels obtain most of the relevant information of the game situation, except speed and direction of the ball. This could be covered by having two consecutive screens. 

\subsection{Preprocessing}
The DeepMind paper use preprocessing where the take the last four screen images, resize them to 84x84 and convert to grayscale with 256 gray levels. This will give $256^{84x84x4} \approx 10^{67970}$ possible game states. This will give $10^{67970}$ rows in the imaginary Q-table - more than the number of atoms in the universe. Many pixel combinations never occur, so possible represent it as a sparse table containing only visited states. Most of the states are rarely visited and it would take a lifetime of the universe to the q-table to converge. It should also be possible to have a good guess for Q-values for states we have never seen. 

To do this deep learning comes in to the picture. Neural networks are extremely good at coming up with good features for highly structured data. A way neural networks can be used in reinforcement learning is it could represent the Q-function, where it takes the state (four game screens) and action as an input and output the corresponding Q-value. Another way is to only take game screens as input, and output the Q-value for each possible action. The second approach has the advantages, that if we want to perform a Q-value update or choose the action with the highest Q-value, it can be done by only doing one forward pass in the network and have all Q-values for all possible actions. The two different approaches to use neural network to represent the Q-function can be seen on \Cref{fig:DQN_two_approach}.              


\begin{figure}[H]
	\centering
	\includegraphics[width=1\textwidth]{Figures/Architecture/DQN/DQN_two_approach.pdf}
	\caption{Left: Naive formulation of deep Q-network. Right: More optimized architecture of deep Q-network, used in DeepMind paper.} 
	\label{fig:DQN_two_approach}
\end{figure}    

The network architecture that DeepMind used can be seen on the table below \Cref{tab:DQN_network}. 

\begin{table}[H]
	\centering
	\caption{Architecture of the Deep Q Network used by DeepMind}
	\label{tab:DQN_network}
	\begin{tabular}{|l|c|c|c|c|c|c|}
		\hline
		\rowcolor[HTML]{9B9B9B} 
		\multicolumn{1}{|c|}{\cellcolor[HTML]{9B9B9B}\textbf{Layer}} & \textbf{Input} & \textbf{Filter size} & \textbf{Stride} & \textbf{Num filters} & \textbf{Activation} & \textbf{Output} \\ \hline
		\cellcolor[HTML]{FFFFFF}\textbf{conv1}                       & 84x84x4        & 8x8                  & 4               & 32                   & ReLU                & 20x20x32        \\ \hline
		\rowcolor[HTML]{C0C0C0} 
		\textbf{conv2}                                               & 20x20x32       & 4x4                  & 2               & 64                   & ReLU                & 9x9x64          \\ \hline
		\cellcolor[HTML]{FFFFFF}\textbf{conv3}                       & 9x9x64         & 3x3                  & 1               & 64                   & ReLU                & 7x7x64          \\ \hline
		\rowcolor[HTML]{C0C0C0} 
		\textbf{fc4}                                                 & 7x7x64         &                      &                 & 512                  & ReLU                & 512             \\ \hline
		\cellcolor[HTML]{FFFFFF}\textbf{fc5}                         & 512            &                      &                 & 18                   & Linear              & 18              \\ \hline
	\end{tabular}
\end{table}


        
\section{Deep Deterministic Policy Gradient (DDPG)}\label{DDPG}
As mentioned in \Cref{sec:DQN} the Deep Q-Network solves problems with high-dimensional observation space. But the problem is it can only handle discrete and low-dimensional action space. Many task of interest, most notably physical control tasks, have continuous (real valued) and high-dimensional action spaces. The problem with the Deep Q-Network is it cannot be applied to continuous domains since it relies on finding the action that maximizes action-value function. In the continuous valued case requires an iterative optimization process at every step. \cite{DBLP:journals/corr/LillicrapHPHETS15}

An obvious approach to adapting the Deep Q-Network method to continuous domain is to simply discretize the action space. This have many limitation, most important the curse of dimensionality - the number of actions increases exponentially with the number of degrees of freedom. An example is the human arm have 7 degrees of freedom, with an assumption discretization $a_i \sim  \{-k,0,k\}$ for each joint leads to an action space with dimensionality: $3^7 = 2187$. This problem just become bigger with a finer discretion. Such a large action space makes it difficult to explore efficiently. Discretization of action spaces throws away information of the action domain. 

The Deep Deterministic Policy Gradient try to solve these problems. The Deep Deterministic Policy Gradient method is a model-free off-policy actor-critic algorithm using deep function approximators that can learn policies in high-dimensional, continuous action space. 

The DDPG algorithm uses some of the same deep learning tricks as the Deep Q-Network (DQN) see \Cref{sec:DQN}. To explain more about this algorithm a car simulation environment called TORCS (The Open Racing Car Simulator) is used see \Cref{sec:TORCS} for more information about the TORCS environment \cite{DDPG_Torcs}.

\subsection{Algorithm}
Even with the DDPG using some of the tricks from the DQN algorithm, it is not straight forward to apply the Q-learning to continuous action space. It is because in the continuous action space finding a greedy policy requires an optimization of \textit{$a_t$} at every time step - this optimization is too slow to be practical with large, unconstrained function approximators and non-trivial action spaces. Here is instead used an actor-critic approach based on the DPG (deterministic policy gradient) algorithm \cite{DBLP:conf/icml/SilverLHDWR14} 

The DPG algorithm use a parameterized actor function $\mu(s|\theta^\mu)$ which specifies the current policy by deterministically mapping states to a specific action. The critic $Q(s,a)$ is learned by using the Bellman equation as in Q-learning. The actor is updated by applying the chain rule to the expected return from the start distribution J with respect to the actor parameters: 
\begin{equation}
\triangledown_{\theta^\mu} J \approx \mathbb{E}_{s_t \sim \rho^\beta} [\triangledown_{\theta^\mu}Q(s,a|\theta^Q)|_{s=s_t , a=\mu(s_t|\theta^\mu)}]  
\newline
\end{equation}
\begin{equation}
\triangledown_{\theta^\mu} J = \mathbb{E}_{s_t \sim \rho^\beta} [\triangledown_{a}Q(s,a|\theta^Q)|_{s=s_t , a=\mu(s_t)} \triangledown_{\theta_\mu}\mu(s|\theta^\mu)|_{s=s_t} ]
\end{equation} 

This was proved that it is the policy gradient - the gradient of the policy performance. 

Introducing non-linear function approximators means that convergence is no longer guaranteed. The approximators is essential to learn and generalize on large state spaces. The DDPG contribution is to provide modification to DPG inspired of the success of the DQN, which allow it to use neural networks function approximators to learn in state and action space online.  

One of the challenges of using neural networks for reinforcement learning is most optimization algorithms assume that the samples are independently and identically distributed. To solve this problem a replay buffer is used, it samples a minibatch uniformly from the buffer - more about the replay buffer see \Cref{sec:DQN}. Because the DDPG is an off-policy algorithm, the replay buffer can be large, allowing the algorithm to benefit from learning across a set of uncorrelated transition. 

\subsubsection{Update weights}
Directly implementing the Q learning with neural networks is unstable in many environments. Since network $Q(s,a|\theta^Q)$ being updated is also used in calculating the target value, the Q update is likely to divergence. It is modified from the DQN algorithm for actor-critic using "soft" target updates, instead of directly updating the weights. This is done by creating a copy of the actor and critic networks, $Q'(s,a|\theta^{Q'})$ and $\mu'(s|\theta^{\mu'})$ respectively, they are used for calculating the target values. The weights of these target networks are then updated by having them slowly track the learned networks: 
\begin{equation}
\theta' \leftarrow \tau \theta + (1-\tau)\theta'   \quad \textrm{with} \quad \tau \ll 1 
\end{equation}   
This means that the target values are constrained to change slowly, greatly improving the stability of learning. This change helps the unstable problem of learning the action-value function closer to the case of supervised learning, where a robust solution exists. The DDPG needs both a target $\mu'$ and $Q'$ was required to have stable targets $y_i$ to consistently train the critic without divergence. It may slow learning, since the target network delays the propagation of value estimations. In practice, it is better because the stability of training is more important than the learning speed. 

\subsubsection{Batch normalization}
Different components of the observation may have different physical units (for example, position versus velocities) and the ranges may change through the different environments. This can make it difficult for the network to learn effectively and may make it difficult to find hyper-parameters which generalize across environments with different scales of state values.

One approach to solve this problem is to manually scale the features so they have similar ranges across different environments and units. The way this problem is solved in the DDPG algorithm is by using a technique for deep learning called batch normalization. This technique normalizes each dimension across the samples in a minibatch to have a unit mean and variance. It maintains a running average of the mean and variance to use for normalization during training, in the DDPG for exploration or evaluation. In a low-dimensional case the batch normalization is used on the state input and all layers of the actor network ($\mu$ network) and all layers of the critic network ($Q$ network) prior to the action input. With batch normalization, the DDPG is able to learn effectively across many different tasks with different types of units, without needing to manually ensure units in different ranges.

\subsubsection{Exploration}
A major challenge of learning continuous action spaces is exploration. An advantage of off-policy algorithm such as DDPG, is that it can treat problems of exploration independently from learning algorithm. The DDPG has an exploration policy $\mu'$ by adding noise sampled from a noise process $N$ to the actor policy: 
\begin{equation}
\mu'(s_t) = \mu(s_t|\theta_t^\mu) + N
\end{equation} 
N can be chosen to suit the environment. The noise can be added using a process called Ornstein-Uhlenbeck to do the exploration.

The DDPG algorithm can be seen on \Cref{algo:DDPG}, it is the algorithm DeepMind uses \cite{DBLP:journals/corr/LillicrapHPHETS15}.  



\begin{algorithm}[H]
	\caption{Deep Deterministic Policy Gradient (DDPG) algorithm}
	\label{algo:DDPG}
	\begin{algorithmic}[H]
		\State Randomly Initialize critic network $Q(s,a|\theta^Q)$ and actor $\mu(s|\theta^\mu)$ with weights $\theta^Q$ and $\theta^\mu$
		\State Initialize target network $Q'$ and $\mu'$ with weights $\theta^{Q'} \leftarrow \theta^{Q}, \theta^{\mu'} \leftarrow \theta^{\mu}$ 
		\State Initialize replay Buffer $R$
		\For {$episode = 1$ to M} 
			\State Initialize a random process $N$ for action exploration
			\State Receive initial observation state $s_1$
			\For {$t = 1$ to T}
				\State Select action $a_t = \mu(s_t|\theta^\mu + N_t)$ according to the current policy and exploration noise
				\State Store transition $(s_t,a_t,r_t,s_{t+1})$ in $R$
				\State Sample random minibatch of $N$ transitions$(s_i,a_i,r_i,s_{i+1})$ from $R$
				\State Set $y_i = r_i+\gamma Q'(s_{i+1},\mu'(s{i+1}|\theta^{\mu'})|\theta^{Q'})$
				\State Update critic by minimizing the loss: $L=\frac{1}{N} \sum_{i}(y_i - Q(s_i,a_i|\theta^Q))^2$
				\State Update the actor policy using the sampled policy gradient:   
		  			   \begin{equation*}
		  			   \triangledown_{\theta^\mu} J = \frac{1}{N} \sum_{i} \triangledown_{a}Q(s,a|\theta^Q)|_{s=s_i , a=\mu(s_i)} \triangledown_{\theta_\mu}\mu(s|\theta^\mu)|_{s=s_i} 
		  			   \end{equation*}
		  		\State Update the target networks:
		  			   \begin{equation*}
		  			   \theta^{Q'} \leftarrow \tau \theta^Q + (1-\tau)\theta^{Q'} 
		  			   \end{equation*}
		  			   \begin{equation*}
		  			   \theta^{\mu'} \leftarrow \tau \theta^\mu + (1-\tau)\theta^{\mu'} 
		  			   \end{equation*}
			\EndFor
		\EndFor
	\end{algorithmic}
\end{algorithm}


\subsection{Network}
To use the DDPG algorithm 3 networks need to be created, an actor network, a critic network and a target network. To understand the structure of these networks, we will use an example from the TORCS environment \cite{DDPG_Torcs}. In this example, the input is the sensor data from the car used as the state of the environment. It has 29 different sensor inputs \cite{Data_from_Torcs}, so the state space has the size of 29.  

Before the actor and critic network are explained, the actor-critic algorithm will be explained. The actor-critic algorithm is a hybrid method which combine the policy gradient method and the value function method. The policy is known as the actor, while the value function is the critic. The actor produces an action $a$ given the current state of the environment $s$. The critic then produces a signal to criticizes the actions made by the actor. To refer to the human world is this a normal behavior, where the junior employee (actor) do the actual job and his boss (critic) criticizes the work and hopefully the junior employee can do it better next time. In the TORCS example, it uses the continuous Q-learning (SARSA) as the critic model and uses policy gradient method as the actor model. The following figure explains the relationships between Value Function, Policy Function and actor-critic algorithm.    

The tricks from the Deep Q-Network is used, where the Q function is replaced with a neural network $Q^\pi(s,a) \approx Q(s,a,w)$ where w is the weight of the neural network. This end up with the formula for Deep Deterministic Policy Gradient: 
\begin{equation}
\frac{\partial L(\theta)}{\partial \theta} =\frac{\partial Q(s,a,w)}{\partial a} \frac{\partial a}{\partial \theta}
\end{equation}
Where the policy parameters $\theta$ can be updated via stochastic gradient ascent.

As in the DQN an iterative method is used to solve the Q-function, where the loss function is:
\begin{equation}
Loss = [r + \gamma Q(s',a') - Q(s,a)]^2
\end{equation} 

The Q-value can be used to estimate the values of the current actor policy. The actor-critic algorithm can be seen on figure \Cref{fig:Actor_critic_architecture}. 

\begin{figure}[H]
	\centering
	\includegraphics[width=0.6\textwidth]{Figures/Architecture/DDPG/Actor_critic_architecture.pdf}
	\caption{Actor-critic algorithm which is used in DDPG}
	\label{fig:Actor_critic_architecture}
\end{figure}

\subsubsection{Actor network}
The Actor network is created using two hidden layers with 300 hidden units and 600 hidden units. The output consists of three continuous actions, Steering, which is a single unit with tanh activation function (where -1 means turn maximum right and +1 means turn maximum left). Acceleration, which is a single unit with sigmoid activation function (where 0 means no gas, 1 means full gas). Brake, another single unit with sigmoid activation function (where 0 means no brake, 1 full brake). The actor network can be seen on \Cref{fig:DDPG_Actor_network}. The hidden layers are all full connected layers. To be able to send the actions to the game the three different actions (steering, acceleration and brake) must be merge to the same action.

\begin{figure}[H]
	\centering
	\includegraphics[width=0.7\textwidth]{Figures/Architecture/DDPG/DDPG_Actor_network.pdf}
	\caption{Actor network used in DDPG for playing TORCS }
	\label{fig:DDPG_Actor_network}
\end{figure}  


\subsubsection{Critic network}
The construction of the Critic Network is very similar to the Deep Q-Network described in \Cref{sec:DQN}. The only difference is that is uses two hidden layers with 300 and 600 hidden units. The critic network takes both the states and the action as inputs. According to the DDPG paper \cite{DBLP:journals/corr/LillicrapHPHETS15}, the actions were not included until the 2’nd hidden layer of Q-network. This critic network can be seen on \Cref{fig:DDPG_Critic_network}. On the figure is a third hidden layer shown, but it is still the second layer, just the state, merged with the actions in the original second layer. 

\begin{figure}[H]
	\centering
	\includegraphics[width=0.7\textwidth]{Figures/Architecture/DDPG/DDPG_Critic_network.pdf}
	\caption{Critic network used in DDPG for playing TORCS }
	\label{fig:DDPG_Critic_network}
\end{figure}  

\subsubsection{Training}
The way the actor and critic network is used for learning how to play TORCS is: 
\begin{enumerate}
	\item The sensor data is received from the environment (TORCS) 
	\item The sensor input will be fed into the neural network, and the network will output 3 real numbers (value of the steering, acceleration and brake)
	\item The network will be trained many times, via the Deep Deterministic Policy Gradient, to maximize the future expected reward.
\end{enumerate}

The actual training is not that complicated. First update the critic by minimizing the loss:
\begin{equation}
L = \frac{1}{N} \sum_{i} (y_i - Q(s_i,a_i|\theta^Q))^2
\end{equation}

Then the actor policy is updated using the sampled policy gradient:
\begin{equation}
\triangledown_\theta J = \frac{\partial Q^\theta(s,a)}{\partial a} \frac{\partial a }{\partial \theta}
\end{equation}

$a$ is the deterministic policy: $a=\mu(s|\theta)$. It can be written as:
\begin{equation}
\triangledown_\theta J = \frac{\partial Q^\theta(s,a)}{\partial a} \frac{\partial \mu(a|\theta) }{\partial \theta}
\end{equation}

The last thing is to update the target network:
\begin{equation}
\theta^{Q'} \leftarrow \tau \theta^Q + (1-\tau)\theta^{Q'} 
\end{equation}
\begin{equation}
\theta^{\mu'} \leftarrow \tau \theta^\mu + (1-\tau)\theta^{\mu'} 
\end{equation}

The results of the DDPG in TORCS" can be found in Ben Lau blog "Using Keras and Deep Deterministic Policy Gradient to play TORCS" \cite{DDPG_Torcs}.
\section{TORCS (The Open Racing Car Simulator)}
\label{sec:TORCS}
The environment to use in reinforcement learning is important, because it is here all the learning will be done. This project is about using reinforcement learning to learn a car how to drive. After reading the post \cite{DDPG_Torcs} that uses TORCS in combination with Gym-TORCS. This project will use the same environment because it seems like it will make a good simulation environment for the task in this project. It is also an environment DeepMind has also used the TORCS environment to test their algorithms \cite{DBLP:journals/corr/LillicrapHPHETS15} and \cite{DBLP:journals/corr/MnihBMGLHSK16}, it is a thereby a well known environment to use in reinforcement learning. 

The Open Racing Car Simulator or TORCS is a highly portable multi platform car racing simulation. It is used as ordinary car racing game, as AI racing game and as research platform. The source code of TORCS is licensed under the GPL ("Open Source") \cite{TORCS_website}. 

Gym-TORCS is the reinforcement learning environment in TORCS domain. It is made so it have an interface witch matched Open-AI - A toolkit for developing and comparing reinforcement learning algorithms. It supports teaching agents everything from walking to playing games like Pong or Go \cite{OPENAI_website}. It is smart to match the Open-AI interface because it combines a lot of environment to solve different reinforcement learning tasks. This is done by a simple interface to make it easier to use. \cite{Gym_TORCS_website}.

Some points why TORCS is useful as an environment for reinforcement learning:
\begin{itemize}
	\item The AI can learn how to drive
	\item It is possible to visualize how the neural networks learn over time and inspect its learning process. Instead of only looking at the final result
	\item It is easy to visualize when the neural network gets stuck in a local minimum.
	\item Gives an understanding of machine learning technique in automated driving, which is important for self-driving car technologies 
\end{itemize}
      
\subsection{Uses in this project}      
The way this environment has been used in this project is by getting information from the TORCS domain. This information could be the gamescreen, so the pixels of the game. Another thing the environment has been used for is to send commands to the game - this could be in what direction the car should turn.

The reinforcement learning should be able to train a network, where the input to the network is the information coming from the game. The output from the network is the determined action, which then will be send to the game. This interaction between the game and the network can be seen on \Cref{fig:TORCS_interaction}.    
 
\begin{figure}[H]
	\centering
	\includegraphics[width=1\textwidth]{Figures/Architecture/TORCS_interaction.pdf}
	\caption{The interaction between the game TORCS) and the network }
	\label{fig:TORCS_interaction}
\end{figure}

 
As seen on \Cref{fig:TORCS_interaction} the game which in this project is TORCS, sends the screen of the game. Then some preprocessing is happening for getting this image to the right format, so the network is able to analyze it.

The first thing which is done in the preprocessing is to make the image to grayscale. This is done by taken the RGB-image array($64 \times 64 \times 3$) and separate the 3 color channels red, green and blue. These RGB-values is converted to grayscale values by forming a weighted sum of the R, G and B components:
\begin{equation}
grayscale = 0.2989 \cdot R + 0.5870 \cdot G + 0.1140 \cdot B 
\end{equation}  

   
\section{Asynchronous Advantage Actor Critic}
The newest breakthrough in RL is the \textit{Asynchronous Advantage Actor Critic} (A3C) approach, and therefore it was chosen to be studied and implemented onto the idea of driver-less cars. In order to achieve a similar environment as the one of a car, and also to be able to get the necessary data easier a car simulator was chosen for the project. Among the existing car simulators encountered in the different online sources, and after analyzing the DDPG implementation on The Open Racing Car Simulator (TORCS) \cite{DDPG_Torcs}, the project settled for this car simulator as well.

The project was mainly inspired from the article \cite{DBLP:journals/corr/MnihBMGLHSK16} summarized in the section \ref{AsyncMeths}, and also from the recent implementation of the A3C into the Doom game elaborated on in \cite{A3CDoom}.

The idea behind the A3C is very much around the same \textit{actor-critic} approach described in the section \ref{PolicyGradMeths} and in the Figure \ref{fig:Actor_critic_architecture}, that, more accurately, is founded on the presence of both the value function approximator, $\theta_{v}$ and the bootstrapping policy estimator, $\theta$. An additional feature to the actor-critic method is the \textit{asynchronous} part. Instead of having a single agent training on the GPU as in the example of the DDPG project elaborated in the previous section, multiple agents are instantiated for training on different CPU threads simultaneously, and, unlike in the DDPG where there are two different networks, in the A3C the agents share a global network, which is updated as the agents advance. Another new feature of the A3C is the \textit{advantage} element, which is just a mathematical way of expressing how much better some actions ended up to be, and where the estimation should be improved. The update performed by the A3C is of the form $\nabla_{\theta'}\textup{log}\pi(a|s,\theta')A(s,a,\theta,\theta_{v})$, and the formula for the advantage is $\sum_{i=0}^{k-1}\gamma^{i}r_{t+i}+\gamma^kV(s_{t+k},\theta_{v})-V(s_{t},\theta_{v})$, which are both taken from the article \cite{DBLP:journals/corr/MnihBMGLHSK16}. The update formula changes slightly after including the entropy factor in the policy in order to encourage exploration and avoid convergence to an earlier suboptimal solution. The detailed A3C algorithm taken from \cite{DBLP:journals/corr/MnihBMGLHSK16} is listed below.
\begin{algorithm}[H]
	\caption{Asynchronous advantage actor-critic - pseudocode for each actor-learner thread.}
	\label{algo:A3C}
	\begin{algorithmic}
		\State \textit{//Assume global shared parameter vectors $\theta$ and $\theta_{v}$ and counter $T=0$}
		\State \textit{//Assume thread-specific parameter vectors $\theta'$ and $\theta_{v}'$}
		\State Initialize thread step counter $t\leftarrow1$
		\Repeat
		\State Reset gradients: $d\theta\leftarrow 0$ and $d\theta_{v}\leftarrow0$
		\State Synchronize thread-specific parameters $\theta'=\theta$ and $\theta_{v}'=\theta_{v}$
		\State $t_{start}=t$
		\State Get state $s_{t}$
		\Repeat
		\State Perform action $a_{t}$ according to policy $\pi(a_{t}|s_{t}, \theta')$
		\State Receive reward $r_{t}$ and new state $s_{t+1}$
		\State $t\leftarrow t+1$
		\State $T\leftarrow T+1$
		\Until terminal \textbf{or} $t-t_{start}==t_{max}$
		\If {$s_{t}$ is terminal}
		\State $R=0$
		\Else 
		\State $R=V(s_{t},\theta_{v}')$ \textit{//bootstrap from last state }
		\EndIf
		\For {$i \in \left \{ t-1,...,t_{start} \right \}$}
		\State $R\leftarrow r_{i}+\gamma R$
		\State Accumulate gradients wrt $\theta'$:
		\State $d\theta\leftarrow d\theta+\nabla_{\theta'}\textup{log}\pi(a_{i}|s_{i},\theta')(R-V(s_{i},\theta_{v}'))$
		\State Accumulate gradients wrt $\theta_{v}'$:
		\State $d\theta_{v}\leftarrow d\theta_{v} + \partial (R-V(s_{i},\theta_{v}'))^2/ \partial\theta_{v}'$
		\EndFor
		\State Perform asynchronous update of $\theta$ using $d\theta$ and of $\theta_{v}$ using $d\theta_{v}$
		\Until $T>T_{max}$
	\end{algorithmic}
\end{algorithm}

\subsection{Project description}
The A3C implementation into the Doom game uses images generated by the vizDoom environment as input data for the well known nonlinear function approximation solution method - ANN. The images become the \textit{states} of the RL problem based on which the AI agent learns to shoot the opponents. The vizDoom environment has a \textit{reward} function predefined, which is triggered when the agent deploys an action in the environment. As the given implementation was made for the \textit{discrete actions space}, the estimated policy or the \textit{actor} provides the probabilities of taking each action in a specific state, and so, in each state the action with the maximal probability is chosen to be pursued. The \textit{critic}, on the other hand, estimates a state-value function for the existing policy and it is used in the composition of the loss function, which represents the performance measure or the \textit{objective function} of the RL problem.

The A3C project developed for the TORCS environment has the same principles as the A3C Doom. It also uses images as the input states into a deep ANN structure and the mathematical model of the problem is similar. The TORCS environment is more flexible and that is an advantage as it offers more freedom for changing things and get better results. For example, it is easier to change the reward function and perform action manipulations, and this will be further explained and illustrated in the next chapter. Another additional implementation is the \textit{continuous} actions space, which is slightly different than the \textit{discrete} actions space implementation; nevertheless they were both preserved in the project for analysis purposes. 

\subsection{ANN structure}\label{ANNstructure}
The structure of the deep ANN will offer more insight on how the program works. Therefore, the overall structure of the ANN of the A3C TORCS project is presented in the following figure:
\begin{figure}[H]
	\centering
	\includegraphics[width=1.25\textwidth]{Figures/A3CTorcs}
	\caption{The ANN structure of the A3C implementation into Torcs}
	\label{A3CTorcs}
\end{figure}
The gray-scale image of size 64x64 comes into the ANN as input and goes through the first layer of ANN - a convolutional layer that outputs 16 feature maps of sizes 8x8 each. Next, these are again passed through another convolutional layer that outputs 32 feature maps of size 4x4 each, while taking care of spatial correlations. Then the output is flattened with a fully connected layer and passed to a recurrent layer - basic LSTM, that takes care of the temporal dependencies. Finally, the output of the LSTM layer can be used for the last layers of the ANN. The value function is linearly estimated, while the discrete policy is estimated with a softmax activation function and gives the probabilities of each action in the discrete set. For the continuous actions space, on the other hand, the discrete policy is replaced by a linear layer that estimates the mean $\mu$, and a layer that estimates the variance $\sigma^2$ with the softplus activation function. These too are then used in the formation of a normal distribution, which is then sampled to get the action to be passed to the environment. The size of the layer is 3 for the policy as it is used for the 3 actions of the environment, namely, steer, acceleration, and brake. In the discrete case, the policy layer output would be an array of size 3 with the probabilities for each action, e.g. $\left [ 0.5, 0.3, 0.1 \right ]$. In the continuous case, the mean and the variance output layers yield a result that looks similar to the output of the discrete policy layer, but the values have a different interpretation.

\subsection{Program Flow}
Putting the A3C theory and the ANN structure described above, a very general illustration of the flow of the program is generated below for creating a better understanding of the whole project:
\begin{figure}[H]
	\centering
	\includegraphics[width=1.25\textwidth]{Figures/Flow}
	\caption{The flow of the A3C Torcs program}
	\label{Flow}
\end{figure}
From the Figure \ref{Flow} it is possible to notice that first, a global network is defined and a number of agents or workers are instantiated to train themselves in their own environment using a copy of the global network. During training, as the first state of the environment is received and passed through all the ANN layers, the worker picks the action with the highest probability given by the output of the discrete policy layer, and then the worker executes the action while the environment returns the next state and the reward. The states keep coming during an episode and the rewards keep accumulating together with the values in an episode buffer. The episode buffer has a specific size, e.g. 100, so that after 100 episodes it becomes full, the global network is updated with the current value estimate - the bootstrap value, and the episode buffer is emptied. Nevertheless, at every step the global network is also updated with the data from the episode buffer without a bootstrap value. The update of the global network happens in a stable way thanks to the episode buffer and it is performed by applying gradients that were computed for a defined loss function.

The loss function is composed of the value loss, policy loss and the entropy loss. The entropy loss is calculated as the logarithm of the discrete policy multiplied by the discrete policy. The value loss is calculated based on the TD error (mentioned in \ref{Temporal Difference}), which is the squared difference between the target value and the estimated value. The policy loss is computed based on the logarithm of the taken actions multiplied by their probabilities and multiplied by the advantages (mentioned earlier). Each of these are assigned specific weights in the total loss result. 

The loss function represents the performance measure, the goal, or the objective function of the RL problem at hand. The gradient of the loss function with respect to the weights of the ANN is calculated and applied using an optimizer. The recommended optimizers are RMSProp and, an evolved form of it - the \textit{Adaptive Moment Estimation} (Adam) optimizer. Usually, the utilized learning rate for these optimizers is 1e-4. The discount factor used in all of the calculations is 0.99.

The next chapter will present the results of training the ANN of the A3C TORCS implementation for different scenarios, including network and environment specific experiments.