\section{Tracks}
In the TORCS environment is it possible to change the tracks the car is racing in. This is done, to see how it will effect the training. The different tracks is also used for testing. It is possible to test if the car has learned how to drive, by training the car on one track, and then testing it on another track. The car should then be able to drive on both track.  

The setup to test this is the best setup as described in the introduction to this \Cref{cha:Result}. The same setup is then used in all the different track the car has been training on, this is done to make sure that the results from the different track have the same parameters on all the tracks. 

In the environment TORCS there are many different tracks. To get the agent to learn how to drive, a simple track was decided. This was done due to more complicated tracks, had more obstacles like bridges, tress and mountains. A problem with these obstacles is if the agent has to turn left for the first time and it sees many tress there. Then the agent could end up learning it has to turn left every time it sees a tree. then it will take longer for the agent to learn how to drive if there is many obstacles it has to take into account. 

Another issue seen on more complicated tracks is the road the car drives on some times changes through the track also the boundaries of the track can change. This will make it complicated from an agent to learn something, and maybe ending up not learning how to drive. The biggest issue is it will also increase the training time, which will be to long. 

The track shouldn't be to simple, it still has to look like a real road, it also need to have some turns both left and right, so it learns how to drive. 

The track which has been used mostly in this project is a simple track, which have the same road type in the whole track. The track also doesn't have obstacles, only grass around the road, and to boundary from the road to the grass is red and white stripes on the whole track. It has both left and right turns, and also just straight forward, the agent then learn all the different scenarios. The track used mostly in this project can be seen on \Cref{fig:track_simple_1}.

\begin{figure}[H]
	\centering
	\includegraphics[width=1\textwidth]{Figures/Result/track_simple_1.pdf}
	\caption{The first track used is called $simple\_1$ made by Hkou}
	\label{fig:track_simple_1}
\end{figure}

To find other tracks to train and test on, it has to look similar to the one on \Cref{fig:track_simple_1}, because as mentioned before it learns from the input image. So if it should be possible to test the trained agent, it has to see some of the same things on the test track as on the training track. In the TORCS environment there are three tracks, which look similar the one on \Cref{fig:track_simple_1} and two others. 

The second track used in this project is a track which looks really similar to the first track, and the name is simple\_2 instead of simple\_1. The only difference from the first track is, it is mirrored. This mean that all the turns are exactly opposite. The track is thereby good for testing, because it has different turns than the first track. The second track can be seen on \Cref{fig:track_simple_2}.

\begin{figure}[H]
	\centering
	\includegraphics[width=1\textwidth]{Figures/Result/track_simple_2.pdf}
	\caption{The second track used is called $simple\_2$ made by Hkou}
	\label{fig:track_simple_2}
\end{figure}

The last track is similar to the two other tracks on \Cref{fig:track_simple_1} and \Cref{fig:track_simple_2}. The only difference is it is more simple, only has left turns. Another difference form the other two tracks is the length it is much longer than the other tracks. It is because of this length it is used to see how the agent learns on a different track with a longer distance. The last used track in this project can be seen on \Cref{fig:track_longstr}. 

\begin{figure}[H]
	\centering
	\includegraphics[width=1\textwidth]{Figures/Result/track_longstr.pdf}
	\caption{The third track used is called $longstr$ made by Hkou}
	\label{fig:track_longstr}
\end{figure}


