\chapter{Conclusion}\label{Conclusion}
This thesis is in the reinforcement learning area. The main purpose is to implement a method to make a car learn how to drive itself. This is done in a car simulation environment. In this project, TORCS has been used as the car simulation environment.  

To do this getting information in the area of reinforcement learning has been an important part of the project. Here the reinforcement learning course by David Silver \cite{RL_course} and the book by R.S. Sutton and A. G. Barto \cite{Sutton} has been important. 

After getting information different methods, which had a big impact on reinforcement learning, was implemented and tested. This was done for getting a feeling of how the different method worked in different environments.  

This project's contribution to the area of reinforcement learning has been working with the A3C method to accomplish the goal of the project. It means the A3C method in a car simulation environment has been implemented and tested. This implementation has been optimized to learn an agent how to drive the car in an acceptable way. Some scenarios tested is continuous vs. discrete actions, different reward functions, optimizers etc. The best A3C method can learn how to drive the car simulator in the TORCS environment.    

After this project, there are many ideas of improvements for future work. These improvements will help the agent to learn faster, be more stable and more flexible. 

Reinforcement learning is a very interesting area of research, especially because it is an area under big development these days. It means new interesting papers in this area is published with a short time interval. The negative side of this is it can be hard to get help and information in the area, but this can also be an interesting challenge.

This project shows an A3C-implementation which learn a car how to drive in a car simulation environment.     
