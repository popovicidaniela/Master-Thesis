
\documentclass[a4paper,11pt,fleqn,oneside,openright]{memoir} 	% Openright open chapters on right page (openany both pages)

% english setup
\usepackage[english]{babel}
\usepackage[utf8]{inputenc}
\usepackage[T1]{fontenc}

%paragraph setup
%\setlength{\parindent}{0pt} % no indent
<<<<<<< HEAD
%\setlength{\parskip}{3mm} % distance with double Enter to paragraph
=======
\setlength{\parskip}{1.5mm} % distance with double Enter to paragraph
>>>>>>> dabb226fb9fcd7fdd64008ed21b17973dc9d5aed

% Figurs
\usepackage{float}
\usepackage{graphicx}
%\usepackage{color}
% define colors 
%\definecolor{lightgray}{gray}{0.75}


\usepackage{transparent}
\graphicspath{{./figures/}}		% search in this path, so you dont need to write the path to every figure
\usepackage{caption}
\usepackage{subcaption}
\usepackage{epstopdf}
\captionsetup[figure]{labelfont={bf,small,sc},textfont={it,small}}			% figuretext and label, format
\captionsetup[subfloat]{labelfont={bf,small},textfont={it,small}}			% subfigurtext and label, format
\captionsetup[table]{labelfont={bf,small,sc},textfont={it,small}}	
%\usepackage[table,xcdraw]{xcolor}

% Matematik 
\usepackage{amsmath}
\numberwithin{equation}{chapter}	% number of equations after chapters
\usepackage{amssymb,amsthm,bm}
\usepackage{mathtools}				% adjustments to amsmath og fx white space ved summer, limits osv.
%\usepackage[]{algorithm2e}
\usepackage{algorithm}
\usepackage[noend]{algpseudocode}
\usepackage{rotating}
\usepackage{multirow}
\usepackage{booktabs}
\newcommand{\pluseq}{\mathrel{+}=}


% physic and natation 
\usepackage{xspace}					% correct spacing after fx own makroer
\usepackage{xparse}					% improve \new­com­mand
\usepackage{physics}				% Gives a lot of useful commands
\usepackage{microtype}				% Makes small, good chages in text
\usepackage{textcomp}				% makes microtype og SIunitx compatible
\usepackage{siunitx} 				% to use of  units and numbers
\sisetup{range-phrase={\text{ to }},		% Sets between numbers in SIrange
	list-final-separator={\text{ and }},	% between last number in SIlist
	list-pair-separator={\text{ and }},		% betweem numbers in SIlist
	detect-all,						% uses "current font" dvs. fx slanted in figurtext
	separate-uncertainty=true,		% +/- insecure (1) eller brackets (0)
	group-digits=false}				% dont write numbers in groups of tree (0)
\sisetup{exponent-product = \cdot}
\DeclareMathOperator*{\argmax}{arg\,max}  % in your preamble
\DeclareMathOperator*{\argmin}{arg\,min}  % in your preamble 

% Tabels
\usepackage{tabularx}								% Tabels
\usepackage{booktabs}

% Code & APPENDIX
\usepackage[draft,english,footnote,nomargin]{fixme} % make notes to changes
\usepackage{listings}								% packages to include code
\usepackage{color}
%\definecolor{dkgreen}{rgb}{0,0.6,0}
%\definecolor{gray}{rgb}{0.5,0.5,0.5}
%\definecolor{mauve}{rgb}{0.58,0,0.82}
%\lstset{frame=tb,		
%	language=Matlab,
%	aboveskip=3mm,
%	belowskip=3mm,
%	showstringspaces=false,
%	columns=flexible,
%	basicstyle={\small\ttfamily},
%	numbers=left,
%	numberstyle=\tiny\color{gray},
%	keywordstyle=\color{blue},
%	commentstyle=\color{dkgreen},
%	stringstyle=\color{mauve},
%	breaklines=true,
%	breakatwhitespace=true
%	tabsize=3
%}
%\usepackage[numbered]{mcode}								% alternativ setup of matlab-code

\definecolor{ggplotGreen}{rgb}{0.56,0.73,0.26}
\definecolor{ggplotRed}{rgb}{0.87,0.29,0.2}
\definecolor{ggplotBlue}{rgb}{0.2,0.54,0.74}
\lstset { %
	language=C++,
	backgroundcolor=\color{black!5}, % set backgroundcolor
	basicstyle=\footnotesize,% basic font setting
	numbers=left, % display line numbers on the left
	commentstyle=\color{ggplotGreen}, % comment color
	keywordstyle=\color{ggplotBlue}, % keyword color
	stringstyle=\color{ggplotRed} % string color
}

% Random packages
\usepackage{lipsum}									% Lorem Ipsum
%\usepackage[table]{xcolor}
\usepackage[pdftex,dvipsnames,table]{xcolor}  % Coloured text etc.
\usepackage[absolute,overlay]{textpos}
%\rowcolors{2}{gray!10}{white}



% References (the order is important!)
\usepackage{varioref}				% use \vref to refference page
\usepackage[numbers,square]{natbib}
\usepackage{hyperref}				% make refferences to hyperlinks
\hypersetup{
	colorlinks=true,
	linkcolor=black,        % Color of internal links
	citecolor=black,          % Color of links to bibliography
	filecolor=magenta,      % Color of file links
	urlcolor=cyan           % Color of external links
}
\usepackage{cleveref}				% write equations/tabel/figurs/etc. infront  of refferences (\cref)
\usepackage{bibentry}


% ¤¤ Bibliography ¤¤ %

\bibliographystyle{IEEEtranN}				% Style of Bibliography
%\bibliographystyle{bibtex/harvard}			% Style of Bibliography


% Font and Layout
\usepackage{lmodern}	% Latin Modern (font)
\raggedbottom			% accept unbalanced columns (They are not the same height)
\usepackage{soul} 		% bigger space between letters than normal (frontpage)
\sodef\an{}{0.2em}{.9em plus.6em}{1em plus.1em minus.1em} % Frontpage font
\newcommand\stext[1]{\an{\scshape#1}} %Frontpage font
\chapterstyle{plain}	% decide how chapternumber/-titel looks
\checkandfixthelayout[nearest]

% Header/Footer
\nouppercaseheads
\makepagestyle{minStil}
\makeevenhead{minStil}{\slshape\leftmark}{}{}
\makeoddhead{minStil}{}{}{\slshape\rightmark}
\makeevenfoot{minStil}{\thepage}{}{}
\makeoddfoot{minStil}{}{}{\thepage}
\makepsmarks{minStil}{
	\createmark{chapter}{left}{nonumber}{}{}}

\pagestyle{minStil}
%\usepackage{fancyhdr}
%\pagestyle{fancy}
%\fancypagestyle{main}{
%	\fancyhf{}
%	\fancyhead[LE,RO]{\slshape \rightmark}
%	\fancyhead[LO,RE]{\slshape \leftmark}
%	\fancyfoot[C]{\thepage}
%	\renewcommand*{\headrulewidth}{0.4pt}
%	\renewcommand*{\footrulewidth}{0pt}
%}

% Indholdsfortegnegelse
\setsecnumdepth{subsection}
\maxtocdepth{subsection}


%-------- own makroer ---------
\renewcommand{\*}{\cdot}			% Multiplication
\renewcommand{\matrix}[1]{\bm{\mathrm{#1}}} % Bold vectors
\renewcommand{\(}{\left(}			% adjustable left brackets
\renewcommand{\)}{\right)}			% adjustable right brackets
%\newcommand{\sub}[1]{_{\text{#1}}}	% Subscripts with text
%\renewcommand{\sup}[1]{^{\text{#1}}}% Superscripts with text
%\newcommand{\inv}{^{-1}}            % Invers
%\renewcommand{\exp}[1]{\text{exp}\( #1 \)} % exp with brackets
%\renewcommand{\ln}[1]{\text{ln}\( #1 \)} % ln with brackets

%\renewcommand{\sin}[2]{				% Sinus with potenses and brackets
%	\ifthenelse{\equal{#1}{1}}{\text{sin}\(#2\)}{\text{sin}^{#1}\(#2\)}}
%\renewcommand{\cos}[2]{				% Cosinus with potenses and brackets
%	\ifthenelse{\equal{#1}{1}}{\text{cos}\(#2\)}{\text{cos}^{#1}\(#2\)}}
%\renewcommand{\tan}[2]{				% Cosinus with potenses and brackets
%	\ifthenelse{\equal{#1}{1}}{\text{tan}\(#2\)}{\text{tan}^{#1}\(#2\)}}

%\newcommand{\diff}[4]{				% 1: "hard" or "soft" afledte, 2: potens, 3: tæller, 4: nævner
%	\ifthenelse{\equal{#1}{d}}{\frac{d^{#2}#3}{d#4^{#2}}}{
%							   \frac{\partial^{#2}#3}{\partial#4^{#2}}}}
\newcommand*\colvec[1]{\begin{pmatrix}#1\end{pmatrix}}		%make a columnvector (with soft brackts) af arbitary length, calls as: \colvec{1\\2\\...\\n}




% make new definition of squareroots
% it renames \sqrt as \oldsqrt
\let\oldsqrt\sqrt
% it defines the new \sqrt in terms of the old one
\def\sqrt{\mathpalette\DHLhksqrt}
\def\DHLhksqrt#1#2{%
	\setbox0=\hbox{$#1\oldsqrt{#2\,}$}\dimen0=\ht0
	\advance\dimen0-0.2\ht0
	\setbox2=\hbox{\vrule height\ht0 depth -\dimen0}%
	{\box0\lower0.4pt\box2}}

\usepackage{verbatimbox}
%\usepackage{unixode} %unicode characters
\usepackage{tikz}
\usetikzlibrary{positioning,matrix}

\usepackage{datatool}
\usepackage[nopostdot,style=super,nonumberlist,toc]{glossaries}
\makeglossaries

\usepackage[many]{tcolorbox}

\usepackage{xargs}                      % Use more than one optional parameter in a new commands
% Todo list

%\usepackage[colorinlistoftodos,prependcaption,textsize=tiny]{todonotes}
%\newcommandx{\unsure}[2][1=]{\todo[linecolor=red,backgroundcolor=red!25,bordercolor=red,#1]{#2}}
%\newcommandx{\change}[2][1=]{\todo[linecolor=blue,backgroundcolor=blue!25,bordercolor=blue,#1]{#2}}
%\newcommandx{\info}[2][1=]{\todo[linecolor=OliveGreen,backgroundcolor=OliveGreen!25,bordercolor=OliveGreen,#1]{#2}}
%\newcommandx{\improvement}[2][1=]{\todo[linecolor=Plum,backgroundcolor=Plum!25,bordercolor=Plum,#1]{#2}}
%\newcommandx{\thiswillnotshow}[2][1=]{\todo[disable,#1]{#2}}
%


%
% Exampls 
% \todo[inline]{The original todo note withouth changed colours.\newline Here's another line.}
% \unsure{I'm unsure about also!}
% \change{Change this!}
% \info{This can help me in chapter seven!}
% \improvement{This really needs to be improved!}
%%


% For the nice plots
\usepackage{fontspec}
\usepackage{pgfplots}
\usepgfplotslibrary{groupplots}
\usepackage{pgf}

\usepackage{tikz}
\usetikzlibrary{external}
%\tikzsetexternalprefix{figures/tikz/}
\tikzexternalize[prefix=figures/tikz/]
\tikzset{external/mode=graphics if exists}

\newlength\figureheight
\newlength\figurewidth

%\usepackage[T1]{fontenc}
%\usepackage{titlesec, blindtext, color}
%\definecolor{gray75}{gray}{0.75}
%\newcommand{\hsp}{\hspace{20pt}}
%\titleformat{\chapter}[hang]{\huge\bfseries}{\thechapter\hsp}{0pt}{\huge\bfseries}